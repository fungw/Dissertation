
While \ac{VANET} has been around since the early 2000s, vehicular networking has not been put into practise until now. With the roll-out of autonomous vehicles worldwide, one of the next steps for full vehicular automation is to allow cars be aware of parking spaces in their relative area. Various research proposals have been explored to capture parking space vacancies. Using parking sensors located on every parking spot is the most straight-forward solution, however the cost of deployment is by far the highest of all the considerable options. Other solutions include utilising city-wide \ac{CCTV} for detection of parking vacancies, while others have explored utilising laser range-finders on taxis circling around the city to detect spaces. This dissertation includes a current state of the art of the various parking sensing technology to date. Additionally, the dissemination of parking data has also been discussed by several researchers. These include considerations for \ac{VANET} based models, while others have explored more centralised versions of parking data administration; where one single system coordinates parking information to all drivers. In larger cities, parking information dissemination may be distributed into districts, each district managing its own set of parking spaces and notifying the relevant users when spaces become available within their respective vicinity.\\