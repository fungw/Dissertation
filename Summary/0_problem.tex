This dissertation focuses on evaluating \ac{VANET} in Dublin. More specifically, the goal is to investigate whether \ac{VANET} can assist with the dissemination of parking data within a small scale city like Dublin. Parking data allows drivers know where vacant parking spaces are at any given time. However, there are potential problems regarding making parking space information public. One of the central concepts for the introduction of smart parking systems is to minimise traffic congestion within an urban area. By making the parking information public, it does not guarantee that traffic congestion and traffic emissions would decrease. Drivers may run into scenarios where they are en-route to a vacant parking spot, only to find that space has been taken seconds before they arrive. There is a need for a system that oversees the parking spaces, coordinates with cars, roadside units, or other traffic systems to take advantage of the real-time parking data fully. Smart parking may be broken down into two distinct areas. One being the utilisation of sensors to locate parking space occupancies. And the other is the dissemination of the sensed parking data to the relevant users. While the dissertation focuses on the dissemination of data within Dublin, various sensory techniques will also be explored.\\