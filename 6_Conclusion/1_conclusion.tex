\section{Reflection}
Smart parking technology has its limits. The cost of deployment for smart parking systems can be substantial, especially when parking sensors are required to be installed in every known parking spot location. Additionally, a system that oversees these sensors is also required for the deployment of a such a system. The undeniable costs of deployment may deter governments from introducing this system into cities. This is part of the motivation behind exploring various parking space sensing technologies to compile an updated list of alternative sensing solutions.

Sensing parking spaces are one side of the coin, the other side is to introduce a robust system to disseminate the data efficiently and effectively. One of the core related works examined in section \ref{sec:data_dissemination} forms the basis of exploring futuristic solutions for data dissemination. Notably, \ac{VANET} is an increasingly popular field. With the recent introduction of autonomous vehicles, it would be highly unlikely that \ac{VANET} would not have a major role in the development of self-driving vehicles in the future.

Concerns are raised regarding the computational power required on \ac{OBU}s of vehicles. This specific field is only starting to mature, as the considerations put forward regarding inter-vehicular communications has not yet been made a reality. Despite this, considerations have already been explored regarding how a \ac{VANET} architecture should be formed. The questions raised is, whether inter-vehicular communications should be solely dedicated to vehicles on the road, or should \ac{RSU}s act as gateways to a public cloud that serves information to vehicles that support networking, or should there be a hybrid cloud model to satisfy both ends as we head into uncharted territories.

As outlined in the introduction to this dissertation, main motivations to smart parking are to minimise drivers from cruising around looking for spaces. From an environmental perspective, this is vital to sustain good air quality in urban districts. The European Union (EU) have provided targets for the year 2020 for all EU members. One of those targets is a 20\% reduction of non-emission trading scheme sector emissions on 2005 levels \citep{Pereira2005OverviewTargets}. However, according to the \ac{EPA} of Ireland \citep{Gas2017EPAProjections}, by 2020, Ireland is projected to be 4\% - 6\% below 2005 levels. This is very low in comparison to the requirement of 20\% by 2020. Although it does not look hopeful for Ireland to reach its targets by then, it would help to contribute towards the goal in any way possible, achieved or not.

This concludes this dissertation. Thanks for reading.