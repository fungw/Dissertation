\section{Motivation}
One of the core motivation behind smart parking is to minimise the amount of drivers driving around looking for a vacant parking spot within a city. By minimising the process of looking for parking spaces, it will have a positive effect on the environment, as there will be less vehicles on the road and less vehicle emissions. This is also advantageous in terms of traffic congestion in an urban setting, with fewer vehicles on the road, traffic levels may be alleviated. In a study by Donald Shoup \citep{Shoup2006CruisingParking}, ``Between 8 and 74 percent of the traffic was cruising for parking, and the average time to find a curb space ranged between 3.5 and 14 min".

Minimising the amount of drivers driving around looking for a vacant parking spot can be achieved in different ways. One of the methods in achieving this is to introduce parking sensors in parking locations. The parking sensors will be able to sense if a parking spot is vacant or not. This parking data may be disseminated to all drivers. However, by surfacing the parking information to drivers, it does not guarantee that a driver will get a parking spot. In other words, drivers will still need to cruise for parking spots. For this reason, smart parking could potentially be improved upon.

Additionally, \citep{Verroios2011ReachingNetworking} research papers have explored the ways in which data dissemination should be configured. The paper proposes three ways in which parking data could be disseminated. 

\ac{VANET} is the introduction of vehicular communications. There are various \ac{VANET} models; vehicle-to-infrastructure (V2I), vehicle-to-device (V2D), vehicle-to-vehicle (V2V) and vehicle-to-grid (V2G). The \ac{VANET} model of interest in this dissertation is vehicle-to-vehicle (V2V). There are various motivations for \ac{VANET}, some of those included are vehicle platooning, and forward collision mitigation. Vehicle platooning involves vehicles sharing their current speed, acceleration and positioning information with other vehicles within their vicinity. By doing so, vehicles may move in a more streamline motion along a street. In this Japanese study conducted in 2008 \citep{Sugiyamal2008TrafficJam}, one of the reasons of traffic build up is due to a small fluctuation in the speed of an individual vehicle. The study was able to recreate a traffic jam that produced a ``shockwave" effect down the street, similar to a traffic jam in the real world. With the introduction of inter-vehicular communication, cars may be able to keep the flow of traffic in constant manner, thus avoiding unnecessary traffic congestions. Forward collision may assist drivers in times of traffic accidents. Information regarding an accident that occurred on a street may be propagated back along the road through vehicles on that street. Vehicles further along the street may be able to make alternative routes to their destinations to alleviate traffic in that area.

As mentioned, the introduction of parking sensors does not guarantee that a driver gets a parking spot. An optimal method of parking data dissemination could yield a suitable solution to this. In recent years, self driving cars have become increasingly popular. Inter-vehicle communications is not a far-fetched idea 


This dissertation involves designing a VANET smart parking system. 

% In recent years, self driving cars have become increasingly popular. With the roll-out of autonomous vehicles testing by various companies worldwide. At the time of writing, autonomous vehicles are able to search for a vacant parking spot within an area via its on-board sensors and cameras. However, autonomous vehicles will still involve the process of cruising around in search of a parking spot. A potential improvement could feature communication between a smart parking system and the autonomous vehicles. This could take the form of vehicle-to-infrastructure or vehicle-to-vehicle communications. Vehicles could communicate with \ac{RSUs} regarding parking spaces within the area. Alternatively, inter-vehicle communications could provide an ad-hoc parking data dissemination network among participating vehicles. The introduction of vehicle communications is known as \ac{VANET}.

% This dissertation involves designing a VANET smart parking system for Dublin City. Additionally, it plans to investigate and evaluate a VANET smart parking system for a small scale city like Dublin. 
