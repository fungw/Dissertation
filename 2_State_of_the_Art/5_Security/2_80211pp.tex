\subsection{802.11pp}
Inter-vehicle communication has also been explored in different forms. IEEE have devised a new protocol, known as 802.11pp, a dedicated protocol for short range communication between vehicles. The goal is provide an international standard for Wireless Access for Vehicular Environment (WAVE). As vehicles cannot tolerate excessive connectivity delays, the protocol focuses on resolving these issues. The technology originates from Dedicated Short Range Communication (DSRC), which is a short to medium range wireless communication channel for automotive use \cite{dsrc}.

The implementation of the 802.11pp protocol mainly focuses on communication between on-board units (OBUs) and road-side units (RSUs). The messages that occur between the two should involve encryption of private data concerning the location and details of the driver. Additionally, certificates should be issued by the vendor, in this case the parking system communicating with drivers, to authenticate the messages between OBUs and RSUs.

In this paper \cite{ucar_security_2016}, it discusses the potential of attacks on the platooning of vehicles that utilise the 802.11pp protocol. Platooning of autonomous vehicles involves communication between cars in order to traverse through road segments. It works by adjusting the speed, acceleration and deceleration variables of each car within a platoon. The paper discusses how packet falsification may be used to alter the acceleration variables. Alternatively, attackers may demonstrate a replay attack, where they eavesdrop on the communication between the vehicles, store them and replay them at another random time interval. This will disrupt the stability of the platoon and is highly undesirable. In future works, it proposes implementing a secure DSRC protocol that reinforces from jamming, membership falsifications and hijacking. One of the proposed methods is to implement a secure key establishment to achieve confidentiality. Additionally, using the authentication of verified members in the platoon to adopt key management and key refresh mechanisms.

The vulnerabilities of 802.11pp have been explored by various papers. Not only could attackers affect platooning vehicles, but in the case of disseminating parking information also. The falsification of parking data is of concern as attackers may track and divert drivers from vacant spaces with malicious intentions in mind.