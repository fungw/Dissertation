\subsection{802.11p}
Inter-vehicular communication protocols have been explored in different forms. IEEE have devised a new protocol, known as 802.11p, a dedicated protocol for short range communication between vehicles. The goal is to provide an international standard for Wireless Access for Vehicular Environment (WAVE). As vehicles cannot tolerate excessive connectivity delays, the protocol focuses on resolving these issues. The technology originates from Dedicated Short Range Communication (DSRC), which is a short to medium range wireless communication channel for automotive use \cite{dsrc}.

The implementation of the 802.11p protocol focuses on communication between \ac{OBU} and \ac{RSU}. In this paper \citep{Panayappan2007VANET-basedAvailability}, it highlights key considerations regarding certificate authentication for vehicular networking. The vehicle manufacturer would be required to assign a certificate to each vehicles' \ac{OBU}. The traffic authority would be required to assign certificates to each \ac{RSU}. The messages transferred between \ac{OBU} and \ac{RSU} will be encrypted by their corresponding private keys. Additionally, they are required to attach their corresponding certificates during message transfers to authenticate one another.

A replay attack involves a malicious user eavesdropping on the connection between an \ac{OBU} and a \ac{RSU} and recording a set of messages between them. A replay attack is initiated when the malicious user replays the set of recorded messages into the channel. As explained in the paper \citep{Panayappan2007VANET-basedAvailability}, this can be avoided by appending a geo-synchronised time obtained from GPS.

In this paper \cite{ucar_security_2016}, it discusses the potential of attacks on the platooning of vehicles that utilise the 802.11p protocol. Platooning of autonomous vehicles involves communication between cars to traverse through road segments. It works by adjusting the speed, acceleration and deceleration variables of each car within a platoon. The paper discusses how packet falsification may be used to alter the acceleration variables. Alternatively, attackers may demonstrate a replay attack. This will disrupt the stability of the platoon and is highly undesirable. In future works, it proposes implementing a secure DSRC protocol that prevents jamming, membership falsifications and hijacking. One of the proposed methods is to implement a secure key establishment to achieve confidentiality. Additionally, it proposes in using the authentication of verified members in the platoon to adopt key management and key refresh mechanisms.

The vulnerabilities of 802.11p have been explored by various papers. Not only could attackers affect platooning vehicles, but also in the case of disseminating parking information. The falsification of parking data is of concern as attackers may track and divert drivers from vacant spaces with malicious intentions.