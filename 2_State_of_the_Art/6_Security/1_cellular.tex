\subsection{Cellular Networks}
\subsubsection{Confidentiality}
In a smart parking system, the participants' locations and information regarding their destinations, are required in order to calculate a route for them to a designated parking spot. Furthermore, the participants' own details may be required upon registering for using the system.

The confidentiality of such data is of importance as the privacy of the users would be exposed otherwise. The front-facing component of a smart parking system would probably take the form of an app on the drivers' phone. Alternatively, it could be a website that users could access on a computer prior to their commute. In the future, the parking systems could also be an embedded system in every vehicle to assist drivers on-the-go. These services could communicate on cellular networks (3G/4G/etc.) as the driver and their vehicle will most likely be travelling and not tethered to one particular spot.

2G systems have fundamental security issues, with its the lack of encryption \citep{simate2013evaluation}, it proves to be a major security risk for data confidentiality. 3G systems has the Authentication and Key Agreement (AKA) protocol, to ensure that users are connected to an intended end-point. AKA is based on a challenge response based mechanism in order to verify its authenticity. Thus, it allows for secure communications between the user and a trusted endpoint. According to \citep{shirbhate2012providing}, 4G security is much harder to implement as it is implemented in an open architecture system. Compared to its predecessor, 3G systems are built in a closed infrastructure, thus bypassing much of the security invulnerabilities. Also with 4G being IP-based, it poses much more security implementation restrictions \citep{park2007survey}.

It is important that the data received by the system of a user is the same as the data that the user sent. If a malicious user were to intercept and change the details of the data between the system and the end user, the attacker could essentially take control of the drivers' recommended route. The impact that this may lead to could be disastrous. Secure communication channels may be used to ensure that only registered users to the system are allowed to communicate with the expected base stations. 

The issues of peer entity authentication could be circumvented as long as the underlying protocols used supply a secure authentication protocol, for example, with 3G, the AKA protocol. This would ensure that the user the end-point is communicating with is the desired user and not of that of an attacker.