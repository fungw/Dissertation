\subsubsection*{\ac{RFID}}
\ac{RFID} is a technology that uses radio waves to read and capture information stored on a tag attached to an object. \ac{RFID} tags do not require a direct line of sight to function. \ac{RFID} can be described with two main components, a \ac{RFID} tag and a \ac{RFID} reader. The tag consists of a microchip that can store data, as well as an antenna to transfer information bidirectionally. A reader emits a signal to \ac{RFID} tags, and tags can respond with the information that they have in their storage \citep{Want2006AnRFID}.

\ac{RFID} tags may be placed on vehicles, with \ac{RFID} readers installed at parking lot entrances and exits. As described in this paper \citep{pala_smart_2007}, tags can be generated for vehicles and stored in a central database system. When vehicles enter the parking lot, the \ac{RFID} reader scans the \ac{RFID} tags and checks whether they are authorised to enter the parking lot. If authorised, the barrier will be raised to allow the vehicle to enter. The same process is performed to allow the vehicle to exit.

From the paper, it is observed that if two or more vehicles enter the parking lot at the same time, the \ac{RFID} readers will not be able to process both vehicles' information correctly. For this reason, it is advised to only process one vehicle at a time.

According to this paper \citep{dokur_embedded_2016}, \ac{RFID} readers are very expensive. However, the use of \ac{RFID} sensors is highly accurate when performed in a controlled environment.