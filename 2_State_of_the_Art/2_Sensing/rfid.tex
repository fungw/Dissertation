\subsubsection*{\ac{RFID}}
\ac{RFID} is a technology that uses radio waves to read and capture information stored on a tag attached to an object. \ac{RFID} tags do not require a direct line of sight to function. \ac{RFID} can be described as two main components, an \ac{RFID} tag and an \ac{RFID} reader. The tag consists of a microchip that can store data, as well as an antenna to transfer information. A reader emits signals to \ac{RFID} tags, and tags respond with the information that they have in their storage \citep{RfidWww.epc-rfid.info}.

\ac{RFID} tags may be placed on vehicles, with \ac{RFID} readers installed at parking lot entrances and exits. As described in this paper \citep{pala_smart_2007}, tags are generated for vehicles and stored in a central database system. When vehicles enter the parking lot, the \ac{RFID} reader reads the \ac{RFID} tags and checks whether they are authorised to enter the parking lot. If authorised, the barrier will be raised, and upon exit, the set up of an auto-payment system could be enforced to bill the drivers. It is observed that if two or more vehicles enter the parking lot at the same time, the \ac{RFID} readers will not be able to process both vehicles' information correctly. For this reason, it is advised to only process one vehicle at a time.