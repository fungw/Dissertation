\subsubsection*{Computer Vision}
Computer vision involves parsing information from images of a video feed. Applications of computer vision range from the detection of defective products on manufacturing assembly lines to facial recognition software. Additionally, computer vision can be applied to the detection of vacant parking spaces. This section includes an analysis of two papers that utilise computer vision to sense parking information.

In this paper \citep{cho_automatic_2016}, it discusses taking advantage of CCTV cameras inside parking lots to guide autonomous vehicles to a parking spot. The system proposes a central computing unit that communicates with the autonomous vehicles. This assumes that a type of vehicular networking is supported by the vehicles. The work was tested within a lab environment with a USB 3.0 camera. The identification of parking spaces is identified by placing a coloured piece of paper in the centre of each parking location. The coloured paper is hidden when the body of a vehicle is in its position. Thus, the camera can identify whether a parking spot is vacant or occupied. However, the paper concludes that dynamic characteristics, such as camera image noise and geometric orientations may impede its use in general parking lot scenarios.

In this paper \citep{amato_deep_2017}, it proposes a deep learning decentralised parking lot occupancy detector. The solution is based on a deep \ac{CNN} designed for use on smart cameras. \ac{CNN} involves the classification of images and recognising persistent attributes of items and objects \citep{Krizhevsky2012ImageNetNetworks}. Smart cameras are defined in the paper as ``cameras capable of processing the acquired images and transmitting just the result to a remote server". The proposed solution can learn where parking spaces are and detect whether a vehicle is parked on it or not. The advantages of designing a decentralised detection system are that it may be deployed to other smart cameras and learn by itself to detect parking space occupancies. This research took place on publicly available parking lot datasets. ``PKLot" and ``Cnrpark-ext" are the two datasets used in this study \citep{2017PKLot, 2017CNR-EXT}. In both datasets, parking areas in different weather conditions are available. With the datasets, the deep learning agent can classify parking spaces and detect their occupancy statuses. This paper concludes that its solution outperforms other existing solutions within the field of parking space occupancy detection with computer vision.

While the initial paper \citep{cho_automatic_2016} discusses the utilisation of CCTV cameras inside parking lots to guide autonomous vehicles, it could be extended to utilising city CCTV cameras to detect parking spaces. However, since the detection method involves a coloured identifier on the parking location, it may prove difficult to do the same with on-street parking locations.

On the analysis of the second paper \citep{amato_deep_2017}, utilisation of a deep learning \ac{CNN} to detect parking space occupancies can be extended to many other smart cameras. This is a very promising solution in assisting parking lots and urban areas equipped with CCTV cameras to detect parking space occupancies.