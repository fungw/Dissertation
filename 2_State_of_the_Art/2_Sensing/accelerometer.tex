\subsubsection*{Accelerometer}
Accelerometers can be found in many smart phones. Accelerometers can be used to detect a phones orientation, motion and rotation. Information regarding a phones' orientation can allow the phones screen to adjust to landscape mode or portrait mode. Accelerometers in mobile phones can also be used to recognise activity of a user \citep{Brezmes2009ActivityPhone}.

PhonePark is a solution that includes a real-time analysis of device mode transitions in order to detect parking space occupancies \citep{xu_real-time_2013}. It works by utilising information of a users' mobile phone. PhonePark proposes three detection methods of parking space occupancies. These are listed below with an overview of how they work.

\begin{enumerate}
    \item \textit{Bluetooth}: This method of detection involves a mobile phone tethered to a vehicles' Bluetooth system. If the device is tethered to the vehicle, then it assumes that the vehicle is being driven. When the Bluetooth disconnects, as the driver walks (10 meters) away from the vehicle, it infers that the vehicle is parked. 
    \item \textit{Transition Models}: Different states are used to classify whether a user is driving, walking or stationary. If the transition sequence of driving to stationary to walking is observed. PhonePark infers that the vehicle is recently parked. The phones accelerometer is used to make estimations as to whether the user is driving, walking or stationary.
    \item \textit{Pay-by-Phone Piggyback}: This method for parking detection with pay-by-phone piggyback is to allow the user to pay through their mobile phone. Upon payment, the user will be asked for their parking space number. This is forwarded to the pay-by-phone and the parking space occupancy is detected in this way.
\end{enumerate}

While the methods of detection are possible, the paper acknowledges that not all drivers have mobile phones. Thus the solution provided by PhonePark will not be possible. Additionally the paper acknowledges GPS errors, Bluetooth pairing difficulties and incorrect transition classifications all contribute to inaccurate information regarding parking space occupancies.