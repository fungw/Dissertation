\subsubsection*{Acoustic}
Acoustic sensors can detect the noise emitted from vehicle engine sounds. Within a parking lot environment, a solution is proposed to utilise acoustic sensors to detect vehicles \citep{Na2009AcousticSystem}. Acoustic sensors are placed in areas within the parking lot. An acoustic localisation algorithm is introduced to narrow the scope of where it estimates the vehicle has parked. In the parking lot scenario, it assumes that cameras are available. Thus, by re-positioning the cameras' viewing angle towards the acoustically localised area, the parking lot system can confirm whether the parking spot has been occupied. 

Acoustic sensors are very sensitive to the environment they operate within. Although this paper \citep{Lee2008IntelligentNetworks} does not directly focus on the analysis of an acoustic sensor parking space detector, it mentions a possible solution that is worth considering. The paper states that each vehicle has its characteristics, vehicle size, magnetic wave pattern and engine sound. With the combination of magnetometers, ultrasonic sensors and acoustic sensors placed at the entrance of a parking lot. It could build a database of the vehicles entering the parking lot, recording the vehicles' body shape, magnetic wave pattern and engine sounds with the above-mentioned sensors. With the placement of sensors throughout the parking lot, it may be possible to track and localise where a particular vehicle has parked, thus supplying information as to where a vehicle has taken up a parking spot.

While acoustic sensors might not be the most accurate method of detecting parking space occupancies due to its sensitivity to its environment, they provide a fascinating insight to the utilisation of non-obvious sensors to seek parking space occupancy rates. In both of the mentioned papers, acoustic sensors do not act alone to obtain parking data. The combination of cameras, magnetometers and ultrasonic sensors are necessary to verify the data obtained from acoustic sensors. 