\section{Smart Parking Models}
\subsubsection{Reservation-Based Models}
Reservation-based systems have also been researched, whereby drivers are able to buy spaces before parking their cars. A central reservation-based system is discussed \citep{2}. The paper is mainly focused on parking lot infrastructure, thus the parking management system may be easier to handle. It devises a solution whereby drivers communicate directly with parking lots in order to obtain information and to reserve spots within the parking lot.

On the other hand, on-street parking reservation-based systems have also been introduced by CrowdPark \citep{8} and a general on-street crowd-sensing system \citep{9}.

However, crowdsensing applications may be the least expensive to implement. In terms of what this essay is concerned with, simulating crowdsensing through the drivers may be difficult to achieve. Thus a simpler model may be possible through the idea of VANETs.

\subsubsection{Demand Based Models}
Alternatively, another form of parking space distributions and in turn, routing mechanisms for drivers is to offer a demand based distribution of parking spaces.

SFPark \citep{12} is a pilot project for smart parking, utilising parking sensors installed into a selected amount of on-street parking spaces. the uniqueness in this system is that it tries to keep around 10-15\% of parking spaces free on a street or block. Using a demand based pricing model, it increases the price for street parking spots if the street is near 85-90\% occupancy. Vice versa, if the street is 85-90\% free, then the charges will be low. This is to ensure that the variance of parking spaces available are distributed among the streets. The price rate factors in the time of day, and if any events are currently taking place in the vicinity of the parking locations.

This model is also of interest to be included into the simulation.